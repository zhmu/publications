\chapter{Personal Evaluation}
\index{personal evaluation}

\section{Learning experiences}

As of any internship, the goal is twofold: for one, my knowledge should be expanded. And not solely study-related knowledge, but also personal knowledge: how to deal with problems in a project, how to talk to people... things of a more social nature. Being part of a true project at a company, so you get an impression how everything is organised and to be a part of it. Starting a project from the beginning to the end.

\subsection{Study-related}

I've learned a lot on Linux kernel internals. As I am mostly a BSD user myself, I had finally gotten the chance to get familiar with Linux.

I learned a lot about filesystems and their implementation within the Linux kernel. Not only the plain design of a filesystem, but also how this is entangled within the kernel itself.

Debugging code is always a challenge; this is especially true for kernel code. There is no sensible debugger, no breakpoints ... all came down to simple print statements and tracing kernel panics. I didn't have much experience in Linux-specific debugging when I started this assignment.

\subsection{Personal}

As for my personal learning experiences, I feel I've now really gotten an impression how a huge company is being organized. My previous internships were at small to medium organizations, therefore this was quite an interesting experience.

Communication is really valued at Philips Research ... in fact, it is a requirement. If you cannot express yourself towards your colleagues, the environment won't be very enjoyable. At the beginning of each day, not only did I have a talk with my company mentor, but also with the entire team.

The latter took place in a so-called \wi{standup meeting} around 9:00. The purpose of this meeting is to inform everyone what you are doing, what went well and what didn't. This is very useful, for people who have knowledge in the area you are struggling in will tell you, so you immediately know who you can ask for help.

People were surprised to see the discussions between Mischa and me. From the start of the project, it was made clear that the both of us had strong opinions and ideas about how things should be done. However, our disagreements were not to annoy each other, but rather to ensure we would achieve the best result on the project. I believe these discussions have really paid off and at the end, I am glad we were able to discuss with each other in such a way.

\section{Improvements}

This section will only focus on the suggested improvements regarding the process. Technical improvements are outlined in the technical report.

As for the reports, even as a lot of time has been reserved for them, it proved hard to get people to review them in time. This was also mostly to a Philips event taking place in June, the \wi{CRE}, which usurps most people's time. I would suggest anyone having an internship at Philips to take this into account.

This internship has proven once more how important good communication is. I am convinced my mentor and I have had good communication, but since we were both in separate buildings, it never really motivated me to walk over for quick questions or comments. This can be seen as a defect on my side.

\section{Working at Philips Research}

While working at Philips Research, I really felt like being part of a team. People all around you are working on different aspects of the program, and occasionally ask you to help them with bugs they keep overlooking, general comments on their code and such. And of course, the reverse is true as well.

I was really impressed by the facilities present, and even more so with the facilities which could be arranged if needed. At previous internships, I was used to working with deprecated hardware previously used by employees who got new machines. However, anything I needed (a faster computer, a harddisk for running tests on, a new screen, a good mouse etc) were mostly just a call to Ad away.

Finally, what really motivated me was the fact that people take you extremely seriously. They consider you part of the team, and will work with you to resolve any issues. At previous internships, I got the impression I was considered just a student, who shouldn't be taken too seriously.
