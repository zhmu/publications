\chapter{Conclusion}
\index{conclusion}

This report provides a complete outline of the design and realization of the filesystem. The Linux Virtual File System has briefly been covered, to represent the reader an overview what a Linux filesystem needs to provide towards the kernel itself. More in-depth information can be found in \cite{UtLK} and \cite{LKD}.

The filesystem itself has also been described. This description covers the raw data structures as well as implementation-specific aspects of the current Linux implementation. This covers both implementation-specific details such as the free list, as well as limitations in the current implementation.

The following improvements could be made:

\begin{itemize}
\index{File Information Table!serializable}
\item Serializable FIT entries \\
Currently, each inode has a fixed size. As we support up to 80 extents per inode, this means we usually waste quite a lot of space. By serializing inodes, we could store inodes far more efficient.
\item More mindful allocation \\
The space allocator will simply grab the first random free block it finds, without checking if an existing file would be better off allocating this block.
\end{itemize}

The filesystem has shown itself to be a stable and efficient filesystem with excellent transfer speeds.
