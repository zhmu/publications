\newcommand{\dodef}[2]{
	\index{{#1}}{#1}&{#2} \\
	\hline
}

\chapter{Definitions and abbreviations}
\label{defs}

\begin{tabularx}{\textwidth}{|l|X|}
\hline
\dodef{ABISS}{Active Block I/O Scheduling System, a scheduling system which can provide guaranteed read- and write I/O performance to applications \cite{ABISS}}
\dodef{dentry}{Directory entry, data structure (include/linux/dcache.h) in the directory cache which contains information about a specific inode within a directory}
\dodef{extent}{An (offset, length) pair}
\dodef{hard link}{A directory entry that directly references an inode. If there are multiple hard links to a single inode and if one of the links is deleted, the remaining links still reference the inode. \cite{TDaIotFOS}}
mmap(2), unlink(2)&These are references to C library functions. The number between parentheses defines the section number in the appropriate manual page \\
\hline
\dodef{inode}{A data structure used by the filesystem to describe a file. The contents of an inode include the file's type and size, the UID of the file's owner, the GID of the directory in which it was created, and a list of the disk blocks and and fragments that make up the file. \cite{TDaIotFOS}}
\dodef{LIMEFS}{Large-file metadata-In-Memory Extend-based File System, the filesystem created during the internship}
\dodef{soft link}{A file whose contents are interpreted as a path name when it is supplied as a component of a path name. Also called a soft link. \cite{TDaIotFOS}}
\dodef{symbolic link}{See soft link}
\end{tabularx}
