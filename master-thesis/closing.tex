\chapter{Closing words}

Throughout the thesis, we have described a complete specification of the SmartPixel system and presented a model of this specification. We have used the mCRL2 toolset in order to validate these models, initially on a per-algorithm basis while later combining all algorithms into a single system. During the analysis of this combined system, it turned out that attempting to analyze such an inherently timed system with a toolset that currently does not support timing is challenging and at times pretty cumbersome.

However, we can conclude that the algorithms as presented in Chapter \ref{sec:operation} are proved to be correct (and can be improved by applying the modifications as outlined in Section \ref{sec:unreliablecomm}). Great effort has been taken to show how models illustrating the behavior of these protocols were constructed piece by piece, with the intention of showing how such a model can be created from solely an informal specification. Later on, these models were combined and the interactions between the various algorithms were studied, something the informal specification was especially vague on.

The result is that all ambiguities of the original specification have been resolved and the resulting specification indeed performs as intended as shown in Chapter \ref{ch:overallsystem} in a fixed $3 \times 3$ network, which we believe to be adequate to show system correctness. Furthermore, the correctness of all individual algorithms have been proved in both $3 \times 3$ and $4 \times 4$ network configurations. During the verification, at least one major design problem in the original specification has been found and resolved, which would have otherwise gone unnoticed. We have attempted a timed analysis, with some success using the UPPAAL toolset even though it was not usable for large specifications. An attempt to convert a timed model to an untimed model was successful, but this did not provide any fruitful results; the mCRL2 toolset was never designed for automatically generated specifications. This becomes especially clear as it seems unable to process the resulting untimed specification.

During the validation process, a lot of problems in the mCRL2 toolset have been discovered. All serious showstopper issues, like the inability to compare negative numbers, have been resolved quite adequately. However, any request to make the toolset more pleasant to use has been dismissed. While the developers are to be commended for the sheer dedication and response times on bugs, it is our belief that especially for bigger specifications of real-life systems such as the SmartPixel system, the toolset would really benefit from more syntactic sugar: for example, while working on the specification that is provided as Appendix \ref{apx:source}, a lot of time was spent analyzing what ultimately was a typo in the parameter list - only a single parameter needed to be changed, yet the syntax requires the entire process parameters to be duplicated, which is quite tedious and error-prone. Another request that was turned down is the ability to generate the process initializer; i.e. if the network is $3 \times 3$, an initializer consisting of $SmartPixel(0) \parallel \dots \parallel SmartPixel(8)$ would automatically be created. Small features such as these would really be beneficial to users of the toolset.

Another aspect that will greatly benefit the toolset is improved documentation. A subset of the reported issues were the result of problems that are known to the specific tool developer, but not to all other developers. While some of these have been added to the `Known Limitations' page on the mCRL2 website, it would be beneficial to synchronize these issues among the mCRL2 developers. An example of such an issue is the inability to generate a statespace file larger than 4GB using the default output format, which has since been added to the website.

It must be noted that the mCRL2 toolset is quite powerful. Even though it has only been compared with UPPAAL during this project, it quickly became evident that mCRL2 is capable of handling much larger models than UPPAAL: an attempt to validate leader election in a $4 \times 4$ grid is feasible in mCRL2, while $3 \times 3$ appeared barely manageable in UPPAAL. Of course, mCRL2 does not consider the notion of time like UPPAAL does, but it is clear to us that mCRL2 handles big specifications much better than UPPAAL currently does; we cannot say whether this is the result of the lack of time in mCRL2.

Finally, we would like to suggest that the mCRL2 toolset be given more functionality regarding timed analysis. Ideally, this would mean standard support for any timed specification, but this is most likely infeasible on the short term. A more realistic approach could be to integrate the LeMans tool as discussed in Section \ref{sec:lemansintro} in the toolset to transform a timed specification into an untimed specification that can be used for analysis, which would still have the advantage of being able to completely reuse mCRL2's powerful model analysis abilities. The ability to transform timed processes to untimed processes is especially useful since analysis by means of monitoring processes by no means solves all possible issues, plus that the construction of such a monitor is a challenge on its own.
