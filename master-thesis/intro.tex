\chapter{Introduction}
\label{sec:intro}

% DevLab is een initiatief van 13 technologiebedrijven uit het midden- en kleinbedrijf.  De bedrijven maken onderdeel uit van de Development Club (kortweg DevClub), die op haar beurt een onderdeel is van de FHI, de brancheorganisatie voor industri�le elektronica. De FHI staat daarmee aan de basis van DevLab.
% In nauwe samenwerking met het netwerk van hoogleraren wordt wetenschappelijk onderzoek uitgevoerd door enkele promovendi. Universitaire en HBO afstudeerders helpen deze promovendi om kennis te vergaren waarmee de twaalf bedirjven op termijn nieuwe producten en diensten genereren.
% De dagelijkse leiding is in handen van twee lectoren, afkomstig uit het HBO

% Dutch Clay is een intelligente klei bestaande uit kleine korrels, ballen of cellen die als klont kunnen functioneren. De mogelijkheden met een dergelijk soort klei zijn zeer groot. Denk aan drie dimensionale digitalisering, of displays als plakband die hun graphicscontroller als gedistribueerd apparaat hebben. Ieder balletje (pixel) is een processor. Een andere toepassing zou een kneedbare processor kunnen zijn. Hoe meer bolletjes, hoe meer rekenkracht. De bolletjes zouden taken kunnen verdelen, zodat berekeningen parallel geschieden, en er een supersnelle computer ontstaat. Maar ook als speelgoed of in medische toepassingen zal Dutch Clay een grote rol kunnen spelen..

DevLab is an organization performing scientific and applied research, where the goal is to provide extra applicable knowledge to participating companies. All of the thirteen participating companies are small- to medium size companies active within the technology sector, and together they are part of Development Club, which in turn is part of FHI, an institution dedicated to industrial electronics. DevLab cooperates with several universities to allow a number of PhD students to perform academic research. These students are in turn assisted by academic and college students in order to obtain knowledge that allows the participating companies to create new products and services.

There are a number of projects being conducted within the DevLab organization, one of which is Dutch Clay: an intelligent form of clay consisting of small balls or cells. There are almost limitless applications imaginable; think of three dimensional image storage: a bag filled with Dutch Clay, in which you insert your arm and instruct the clay to remember their configuration. Subsequently removing your arm from the bag, shaking the bag a few times in order to mix the clay and instructing the clay to restore the configuration would result in the clay displaying the exact same arm you previously inserted. Or what about using the clay as a distributed system, which implies adding more clay would increase the processing power. 

One of the main challenges in the Dutch Clay system is determining the position of the clay. This has been researched in great detail and this problem currently seems unfeasible in an energy and space-constrained application such as this one. To this end, the clay cells were replaced with cubes in an attempt to simply determining the position of the clay. It turned out that this simplification was not enough to make the system feasible, so subsequently, another simplification was introduced: the problem was reduced to 2D by only considering squares, which are called \emph{SmartPixels}, each connected with up to four adjacent SmartPixels. A network of these SmartPixels can then display predefined figures. The complete overview of the desired functionality is described in Section \ref{sec:overview}.

A current sketch of the system has been provided in \cite{hendriksen08sp}. This sketch discusses a number of states and conditions which need to hold to perform state transitions. Each state performs a series of actions, which are described in an algorithm. However, these descriptions are very superficial and based on pure intuition: no attempts have been made to verify whether a system based on this sketch will behave as intended. This is the overall goal of the assignment: use formal modeling methods to describe such a system, in order to determine whether the algorithms described in the sketch work as intended. A second goal is to determine if and how this system can be improved: once a model of the system has been constructed, it can be used to gain more insight in the operation of the system. Using such insights, there may be possible improvements. Finally, the thesis should describe the model in such detail that readers are able to update the model if they alter the system to see if desired properties still hold. A concrete list of research questions is presented in Section \ref{sec:questions}.
