\chapter{Gebruikte software}
\index{Gebruikte software}
\label{software}

Hieronder volgt een overzicht van alle gebruikte software, in alfabetische volgorde:

\begin{itemize}
\item ActiveState \index{Perl}Perl (http://www.activestate.com) \\
Perl is een bekende scripttaal. Deze is gebruikt om snel scripts te bouwen om dingen te controleren en zaken om te zetten.
\item \index{dia}dia (http://www.gnome.org/projects/dia/) \\
Dia is een open source diagram editor, die ook onder Win32 werkt. Deze is zeer eenvoudig in het gebruik en kan gewoon als \index{PNG}PNG-files diagrammen exporteren.
\item \index{doxygen}Doxygen (http://doxygen.org) \\
Doxygen is een programma dat aan de hand van commentaar in de sourcecode documentatie van functies en klassen en dergelijke kan genereren. Ik heb het optionele GraphViz (http://www.research.att.com/ sw/tools/graphviz/) programma ook ge\"installeerd, zodat het ook netjes grafische diagrammen genereerde van de klassen.
\item \index{GIMP}GIMP (http:///gimp.org) \\
The GIMP (GNU Image Manipulation Program) is een opensource programma waarmee plaatjes eenvoudig bewerkt kunnen worden. Dit is dan ook gebruikt voor de (zeer minimale) aanpassingen van illustraties in de documentatie.
\item libxml2 (http://xmlsoft.org) \\
Deze open source bibliotheek is oorspronkelijk ontwikkeld voor het Gnome project\footnote{www.gnome.org, een desktop omgeving voor *NIX}. Gelukkig is hij buiten Gnome ook bruikbaar als een stabiele en snelle XML parser. Verder is ook de XSLT processor hiervan, xsltproc.exe, gebruikt om door middel van XSLT de definities om te zetten.
\item \index{MikTeX}MikTeX (http://www.miktex.org) \\
Dit is een Win32 \index{LaTeX}\LaTeX \mbox{ }typesetting programma. Het is gebruikt voor het Plan van Aanpak en dit stageverslag, maar ook de PDF van de gedefinieerde commando's.
\item Microsoft HTML Help Workshop \\
Dit programma is gebruikt om van de HTML uitvoer een CHM helpfile te maken.
\item Microsoft Office 2000 \\
Aangezien niemand bij Delem \LaTeX \mbox{ }gebruikt en het de bedoeling was om de architectuur documenten te kunnen onderhouden, is er voor gekozen dit in standaard Word te doen, dat onderdeel van Microsoft Office is.
\item Microsoft Visual Studio .NET 2003 \\
Deze welbekende C++ omgeving is gebruikt om het programma in te implementeren, met behulp van MFC\footnote{Microsoft Foundation Classes}.
\item Merant Version Manager\footnote{Voorheen bekend onder de naam Primavera Version Manager, deze naam werd nog in het plan van aanpak gebruikt} \\
Dit programma is gebruikt als versie beheer systeem. Alle broncode en documenten staan hierin opgeslagen, evenals oudere versies.
\item Merant Tracker \\
Dit programma is gebruikt om PR-en en CR-en bij te houden.
\item \index{vim}vim (http://www.vim.org) \\
VIM is VI iMproved, een open source editor die erg op vi\footnote{Visual edItor, een zeer standaard UNIX editor} lijkt, maar een hele hoop verbeteringen heeft.
\item \index{Quintessential CD v1.27}Quintessential CD v1.27 (http://www.quinnware.com) \\
Deze Win32 CD speler is erg compact, en vermaakte mij tijdens het uitoefenen van mijn stageactiviteiten.
\item \index{zsh}zsh (http://www.zsh.org) \\
De Z Shell is een open source shell, die werkt op *NIX en Win32 machines. Deze is veel krachtiger dan de standaard Microsoft shell en werd daarom gebruikt in plaats van de laatste.
\end{itemize}
