\begin{longtable}{|l|X|}
\hline
\index{API}API&Application Programming Interface, bibliotheekfuncties bedoeld voor communicatie met systeemfuncties.\\
\hline
\index{Besturing}Besturing&Onderdeel van de drukpers dat de aansturing van de modulen regelt.\\
\hline
\index{CAN}CAN&Controller Area Network, orgineel door Siemens ontwikkelde bus voor betrouwbare datacommunicatie.\\
\hline
\index{CCB}CCB&Change Control Board, zie pagina \pageref{CCB} voor meer informatie.\\
\hline
\index{CR}CR&Change Request, verzoek om iets in de software te implementeren/veranderen.\\
\hline
\index{CVS}CVS&Concurrent Version System, uitbreiding op RCS.\\
\hline
\index{Deadlock}Deadlock&Toestand waarin twee of meer stukken code op de zelfde hulpbron wachten, zodat ze elkaar blokkeren.\\
\hline
\index{HSB}HSB&Hoge Snelheid Bus, zie CAN.\\
\hline
\index{LaTeX}\LaTeX&Een typesetting systeem, waarmee boeken en dergelijke gemaakt kunnen worden door middel van een markup taal \\
\hline
\index{MFC}MFC&Microsoft Foundation Classes, API van Microsoft om in C++ (redelijk) eenvoudig Windows programma's te maken.\\
\hline
\index{Module}Module&Regelt de aansturing van (onder andere) motoren. Zie pagina \pageref{module} voor meer informatie.\\
\hline
\index{Milestones}Milestone&Mijlpaal, periode in de fasering. Zie pagina \pageref{fasering} voor meer informatie.\\
\hline
\index{Mutex}Mutex&Afkorting van mutal exclusion, zorgt ervoor dat een hulpbron niet door meerdere stukken code tegenlijk gebruikt kan worden.\\
\hline
\index{Increments}Increment&Periode van 2 weken. Zie pagina \pageref{proces} voor meer informatie.\\
\hline
\index{PDF}PDF&Project Definition, vergelijkbaar met een Plan van Aanpak.\\
\hline
\index{PR}PR&Problem Report, verzoek om een fout in de software te verbeteren.\\
\hline
\index{RCS}RCS&Revision Control System, systeem om versies van tekstbestanden (bijvoorbeeld broncode) te archiveren.\\
\hline
\index{UML}UML&Universal Modelling Language, notatie gebruikt om objectklassen te beschrijven.\\
\hline
\index{URD}URD&User Requirements Document, beschrijft de initi\"ele bedoeling van een project.\\
\hline
\index{XML}XML&eXtensible Markup Language, taal om data in te beschrijven die eenvoudig uitbreidbaar is.\\
\hline
\index{XSLT}XSLT&eXtensible Stylesheet Language Transformations, taal waarmee XML bestanden omgezet kunnen worden naar tekstbestanden met een mogelijk geheel andere opbouw.\\
\hline
\end{longtable}
